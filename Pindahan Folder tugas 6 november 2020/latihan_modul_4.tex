% Options for packages loaded elsewhere
\PassOptionsToPackage{unicode}{hyperref}
\PassOptionsToPackage{hyphens}{url}
%
\documentclass[
]{article}
\usepackage{lmodern}
\usepackage{amssymb,amsmath}
\usepackage{ifxetex,ifluatex}
\ifnum 0\ifxetex 1\fi\ifluatex 1\fi=0 % if pdftex
  \usepackage[T1]{fontenc}
  \usepackage[utf8]{inputenc}
  \usepackage{textcomp} % provide euro and other symbols
\else % if luatex or xetex
  \usepackage{unicode-math}
  \defaultfontfeatures{Scale=MatchLowercase}
  \defaultfontfeatures[\rmfamily]{Ligatures=TeX,Scale=1}
\fi
% Use upquote if available, for straight quotes in verbatim environments
\IfFileExists{upquote.sty}{\usepackage{upquote}}{}
\IfFileExists{microtype.sty}{% use microtype if available
  \usepackage[]{microtype}
  \UseMicrotypeSet[protrusion]{basicmath} % disable protrusion for tt fonts
}{}
\makeatletter
\@ifundefined{KOMAClassName}{% if non-KOMA class
  \IfFileExists{parskip.sty}{%
    \usepackage{parskip}
  }{% else
    \setlength{\parindent}{0pt}
    \setlength{\parskip}{6pt plus 2pt minus 1pt}}
}{% if KOMA class
  \KOMAoptions{parskip=half}}
\makeatother
\usepackage{xcolor}
\IfFileExists{xurl.sty}{\usepackage{xurl}}{} % add URL line breaks if available
\IfFileExists{bookmark.sty}{\usepackage{bookmark}}{\usepackage{hyperref}}
\hypersetup{
  pdftitle={Latihan Modul 4},
  pdfauthor={Nicholas Nanda S},
  hidelinks,
  pdfcreator={LaTeX via pandoc}}
\urlstyle{same} % disable monospaced font for URLs
\usepackage[margin=1in]{geometry}
\usepackage{color}
\usepackage{fancyvrb}
\newcommand{\VerbBar}{|}
\newcommand{\VERB}{\Verb[commandchars=\\\{\}]}
\DefineVerbatimEnvironment{Highlighting}{Verbatim}{commandchars=\\\{\}}
% Add ',fontsize=\small' for more characters per line
\usepackage{framed}
\definecolor{shadecolor}{RGB}{248,248,248}
\newenvironment{Shaded}{\begin{snugshade}}{\end{snugshade}}
\newcommand{\AlertTok}[1]{\textcolor[rgb]{0.94,0.16,0.16}{#1}}
\newcommand{\AnnotationTok}[1]{\textcolor[rgb]{0.56,0.35,0.01}{\textbf{\textit{#1}}}}
\newcommand{\AttributeTok}[1]{\textcolor[rgb]{0.77,0.63,0.00}{#1}}
\newcommand{\BaseNTok}[1]{\textcolor[rgb]{0.00,0.00,0.81}{#1}}
\newcommand{\BuiltInTok}[1]{#1}
\newcommand{\CharTok}[1]{\textcolor[rgb]{0.31,0.60,0.02}{#1}}
\newcommand{\CommentTok}[1]{\textcolor[rgb]{0.56,0.35,0.01}{\textit{#1}}}
\newcommand{\CommentVarTok}[1]{\textcolor[rgb]{0.56,0.35,0.01}{\textbf{\textit{#1}}}}
\newcommand{\ConstantTok}[1]{\textcolor[rgb]{0.00,0.00,0.00}{#1}}
\newcommand{\ControlFlowTok}[1]{\textcolor[rgb]{0.13,0.29,0.53}{\textbf{#1}}}
\newcommand{\DataTypeTok}[1]{\textcolor[rgb]{0.13,0.29,0.53}{#1}}
\newcommand{\DecValTok}[1]{\textcolor[rgb]{0.00,0.00,0.81}{#1}}
\newcommand{\DocumentationTok}[1]{\textcolor[rgb]{0.56,0.35,0.01}{\textbf{\textit{#1}}}}
\newcommand{\ErrorTok}[1]{\textcolor[rgb]{0.64,0.00,0.00}{\textbf{#1}}}
\newcommand{\ExtensionTok}[1]{#1}
\newcommand{\FloatTok}[1]{\textcolor[rgb]{0.00,0.00,0.81}{#1}}
\newcommand{\FunctionTok}[1]{\textcolor[rgb]{0.00,0.00,0.00}{#1}}
\newcommand{\ImportTok}[1]{#1}
\newcommand{\InformationTok}[1]{\textcolor[rgb]{0.56,0.35,0.01}{\textbf{\textit{#1}}}}
\newcommand{\KeywordTok}[1]{\textcolor[rgb]{0.13,0.29,0.53}{\textbf{#1}}}
\newcommand{\NormalTok}[1]{#1}
\newcommand{\OperatorTok}[1]{\textcolor[rgb]{0.81,0.36,0.00}{\textbf{#1}}}
\newcommand{\OtherTok}[1]{\textcolor[rgb]{0.56,0.35,0.01}{#1}}
\newcommand{\PreprocessorTok}[1]{\textcolor[rgb]{0.56,0.35,0.01}{\textit{#1}}}
\newcommand{\RegionMarkerTok}[1]{#1}
\newcommand{\SpecialCharTok}[1]{\textcolor[rgb]{0.00,0.00,0.00}{#1}}
\newcommand{\SpecialStringTok}[1]{\textcolor[rgb]{0.31,0.60,0.02}{#1}}
\newcommand{\StringTok}[1]{\textcolor[rgb]{0.31,0.60,0.02}{#1}}
\newcommand{\VariableTok}[1]{\textcolor[rgb]{0.00,0.00,0.00}{#1}}
\newcommand{\VerbatimStringTok}[1]{\textcolor[rgb]{0.31,0.60,0.02}{#1}}
\newcommand{\WarningTok}[1]{\textcolor[rgb]{0.56,0.35,0.01}{\textbf{\textit{#1}}}}
\usepackage{graphicx,grffile}
\makeatletter
\def\maxwidth{\ifdim\Gin@nat@width>\linewidth\linewidth\else\Gin@nat@width\fi}
\def\maxheight{\ifdim\Gin@nat@height>\textheight\textheight\else\Gin@nat@height\fi}
\makeatother
% Scale images if necessary, so that they will not overflow the page
% margins by default, and it is still possible to overwrite the defaults
% using explicit options in \includegraphics[width, height, ...]{}
\setkeys{Gin}{width=\maxwidth,height=\maxheight,keepaspectratio}
% Set default figure placement to htbp
\makeatletter
\def\fps@figure{htbp}
\makeatother
\setlength{\emergencystretch}{3em} % prevent overfull lines
\providecommand{\tightlist}{%
  \setlength{\itemsep}{0pt}\setlength{\parskip}{0pt}}
\setcounter{secnumdepth}{-\maxdimen} % remove section numbering

\title{Latihan Modul 4}
\author{Nicholas Nanda S}
\date{11/3/2020}

\begin{document}
\maketitle

\hypertarget{deskripsi}{%
\subsection{Deskripsi}\label{deskripsi}}

Ini adalah dokumen R markdown, dibuat untuk menyelesaikan tugas
praktikum data science latihan modul 4.

\hypertarget{import-library-dan-dataset}{%
\subsection{Import Library dan
Dataset}\label{import-library-dan-dataset}}

\begin{Shaded}
\begin{Highlighting}[]
\KeywordTok{library}\NormalTok{(dslabs)}
\KeywordTok{data}\NormalTok{(}\StringTok{"murders"}\NormalTok{)}
\end{Highlighting}
\end{Shaded}

\hypertarget{assign-pop-value}{%
\subsection{1. Assign pop value}\label{assign-pop-value}}

Menyimpan data populasi pada variable pop

\begin{Shaded}
\begin{Highlighting}[]
\NormalTok{pop <-}\StringTok{ }\NormalTok{murders}\OperatorTok{$}\NormalTok{population}
\NormalTok{pop}
\end{Highlighting}
\end{Shaded}

\begin{verbatim}
##  [1]  4779736   710231  6392017  2915918 37253956  5029196  3574097   897934
##  [9]   601723 19687653  9920000  1360301  1567582 12830632  6483802  3046355
## [17]  2853118  4339367  4533372  1328361  5773552  6547629  9883640  5303925
## [25]  2967297  5988927   989415  1826341  2700551  1316470  8791894  2059179
## [33] 19378102  9535483   672591 11536504  3751351  3831074 12702379  1052567
## [41]  4625364   814180  6346105 25145561  2763885   625741  8001024  6724540
## [49]  1852994  5686986   563626
\end{verbatim}

Sorting data populasi

\begin{Shaded}
\begin{Highlighting}[]
\NormalTok{popSort <-}\StringTok{ }\KeywordTok{sort}\NormalTok{(pop)}
\end{Highlighting}
\end{Shaded}

Nilai populasi terkecil

\begin{Shaded}
\begin{Highlighting}[]
\NormalTok{popSort[}\DecValTok{1}\NormalTok{]}
\end{Highlighting}
\end{Shaded}

\begin{verbatim}
## [1] 563626
\end{verbatim}

\hypertarget{indeks-populasi-terkecil}{%
\subsection{2. Indeks populasi
terkecil}\label{indeks-populasi-terkecil}}

Menampilkan indeks tiap-tiap data populasi dan terurut dari yang
terkecil

\begin{Shaded}
\begin{Highlighting}[]
\NormalTok{pop}
\end{Highlighting}
\end{Shaded}

\begin{verbatim}
##  [1]  4779736   710231  6392017  2915918 37253956  5029196  3574097   897934
##  [9]   601723 19687653  9920000  1360301  1567582 12830632  6483802  3046355
## [17]  2853118  4339367  4533372  1328361  5773552  6547629  9883640  5303925
## [25]  2967297  5988927   989415  1826341  2700551  1316470  8791894  2059179
## [33] 19378102  9535483   672591 11536504  3751351  3831074 12702379  1052567
## [41]  4625364   814180  6346105 25145561  2763885   625741  8001024  6724540
## [49]  1852994  5686986   563626
\end{verbatim}

\begin{Shaded}
\begin{Highlighting}[]
\KeywordTok{order}\NormalTok{(pop)}
\end{Highlighting}
\end{Shaded}

\begin{verbatim}
##  [1] 51  9 46 35  2 42  8 27 40 30 20 12 13 28 49 32 29 45 17  4 25 16  7 37 38
## [26] 18 19 41  1  6 24 50 21 26 43  3 15 22 48 47 31 34 23 11 36 39 14 33 10 44
## [51]  5
\end{verbatim}

\hypertarget{fungsi-which.min}{%
\subsection{3. Fungsi which.min}\label{fungsi-which.min}}

Dengan menggunakkan fungsi which.min untuk membuat hasil seperti
sebelumnya.

\begin{Shaded}
\begin{Highlighting}[]
\NormalTok{minMurder <-}\StringTok{ }\KeywordTok{which.min}\NormalTok{(murders}\OperatorTok{$}\NormalTok{population)}
\end{Highlighting}
\end{Shaded}

\hypertarget{nama-negara-dengan-populasi-terkecil}{%
\subsection{4. Nama Negara dengan Populasi
terkecil}\label{nama-negara-dengan-populasi-terkecil}}

\begin{Shaded}
\begin{Highlighting}[]
\NormalTok{murders}\OperatorTok{$}\NormalTok{state[minMurder]}
\end{Highlighting}
\end{Shaded}

\begin{verbatim}
## [1] "Wyoming"
\end{verbatim}

\hypertarget{peringkat-populasi-negara}{%
\subsection{5. Peringkat Populasi
Negara}\label{peringkat-populasi-negara}}

\begin{Shaded}
\begin{Highlighting}[]
\NormalTok{ranks <-}\StringTok{ }\KeywordTok{rank}\NormalTok{(murders}\OperatorTok{$}\NormalTok{population)}
\NormalTok{my_df <-}\StringTok{ }\KeywordTok{data.frame}\NormalTok{(}\DataTypeTok{Nama =}\NormalTok{ murders}\OperatorTok{$}\NormalTok{state, }\DataTypeTok{Ranking =}\NormalTok{ ranks)}
\KeywordTok{head}\NormalTok{(my_df)}
\end{Highlighting}
\end{Shaded}

\begin{verbatim}
##         Nama Ranking
## 1    Alabama      29
## 2     Alaska       5
## 3    Arizona      36
## 4   Arkansas      20
## 5 California      51
## 6   Colorado      30
\end{verbatim}

\hypertarget{peringkat-populasi-negara-terurut-terkecil}{%
\subsection{6. Peringkat Populasi Negara (terurut
terkecil)}\label{peringkat-populasi-negara-terurut-terkecil}}

Mengulangi langkah sebelumnya untuk mengurutkan populasi negara dari
yang terkecil

\begin{Shaded}
\begin{Highlighting}[]
\NormalTok{ranks <-}\StringTok{ }\KeywordTok{rank}\NormalTok{(murders}\OperatorTok{$}\NormalTok{population)}
\NormalTok{my_df <-}\StringTok{ }\KeywordTok{data.frame}\NormalTok{(}\DataTypeTok{Nama =}\NormalTok{ murders}\OperatorTok{$}\NormalTok{state, }\DataTypeTok{Ranking =}\NormalTok{ ranks)}
\NormalTok{ind <-}\StringTok{ }\KeywordTok{order}\NormalTok{(my_df}\OperatorTok{$}\NormalTok{Ranking)}
\NormalTok{my_df}\OperatorTok{$}\NormalTok{Nama[ind]}
\end{Highlighting}
\end{Shaded}

\begin{verbatim}
##  [1] "Wyoming"              "District of Columbia" "Vermont"             
##  [4] "North Dakota"         "Alaska"               "South Dakota"        
##  [7] "Delaware"             "Montana"              "Rhode Island"        
## [10] "New Hampshire"        "Maine"                "Hawaii"              
## [13] "Idaho"                "Nebraska"             "West Virginia"       
## [16] "New Mexico"           "Nevada"               "Utah"                
## [19] "Kansas"               "Arkansas"             "Mississippi"         
## [22] "Iowa"                 "Connecticut"          "Oklahoma"            
## [25] "Oregon"               "Kentucky"             "Louisiana"           
## [28] "South Carolina"       "Alabama"              "Colorado"            
## [31] "Minnesota"            "Wisconsin"            "Maryland"            
## [34] "Missouri"             "Tennessee"            "Arizona"             
## [37] "Indiana"              "Massachusetts"        "Washington"          
## [40] "Virginia"             "New Jersey"           "North Carolina"      
## [43] "Michigan"             "Georgia"              "Ohio"                
## [46] "Pennsylvania"         "Illinois"             "New York"            
## [49] "Florida"              "Texas"                "California"
\end{verbatim}

\newpage

\hypertarget{visualisasi-menggunakan-plot}{%
\subsection{7. Visualisasi menggunakan
Plot}\label{visualisasi-menggunakan-plot}}

Analisis total pembunuhan dengan jumlah populasi.

\begin{Shaded}
\begin{Highlighting}[]
\NormalTok{population_in_millions <-}\StringTok{ }\NormalTok{murders}\OperatorTok{$}\NormalTok{population}\OperatorTok{/}\DecValTok{10}\OperatorTok{^}\DecValTok{6} 
\NormalTok{total_gun_murders <-}\StringTok{ }\NormalTok{murders}\OperatorTok{$}\NormalTok{total }
\KeywordTok{plot}\NormalTok{(}\KeywordTok{log10}\NormalTok{(population_in_millions), total_gun_murders)}
\end{Highlighting}
\end{Shaded}

\includegraphics{latihan_modul_4_files/figure-latex/ploting-1.pdf}

\newpage

\hypertarget{histogram-populasi-negara-bagian}{%
\subsection{8. Histogram Populasi Negara
Bagian}\label{histogram-populasi-negara-bagian}}

\begin{Shaded}
\begin{Highlighting}[]
\NormalTok{populasi <-}\StringTok{ }\KeywordTok{with}\NormalTok{(murders, murders}\OperatorTok{$}\NormalTok{population}\OperatorTok{/}\DecValTok{10}\OperatorTok{^}\DecValTok{6}\NormalTok{)}
\KeywordTok{hist}\NormalTok{(populasi, }\DataTypeTok{main =} \StringTok{"Populasi Negara Bagian"}\NormalTok{)}
\end{Highlighting}
\end{Shaded}

\includegraphics{latihan_modul_4_files/figure-latex/histogram-1.pdf}

\newpage

\hypertarget{boxplot-populasi-negara-bagian-wilayah}{%
\subsection{9. Boxplot Populasi Negara Bagian/
wilayah}\label{boxplot-populasi-negara-bagian-wilayah}}

\begin{Shaded}
\begin{Highlighting}[]
\KeywordTok{boxplot}\NormalTok{(population}\OperatorTok{~}\NormalTok{region, }\DataTypeTok{data =}\NormalTok{ murders)}
\end{Highlighting}
\end{Shaded}

\includegraphics{latihan_modul_4_files/figure-latex/boxplot-1.pdf}

\end{document}
